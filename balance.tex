\documentclass[11pt]{article}
% decent example of doing mathematics and proofs in LaTeX.
% An Incredible degree of information can be found at
% http://en.wikibooks.org/wiki/LaTeX/Mathematics

% Use wide margins, but not quite so wide as fullpage.sty
\marginparwidth 0.5in
\oddsidemargin 0.25in
\evensidemargin 0.25in
\marginparsep 0.25in
\topmargin 0.25in
\textwidth 6in \textheight 8 in
% That's about enough definitions

\usepackage{amsmath}
\usepackage{upgreek}
\usepackage{mathtools}
\usepackage{hyperref}

\begin{document}
\author{Boyan Zhang}
% \title{Homework Set 1: 2.003 Dynamics and Control}
\title{Model Development}
\maketitle

\section{Equations of Motion}

\begin{enumerate}

\item % Problem 1:
Roll equation

Let $\psi$ be steering angle around steering axis
\begin{align}
    M_{\chi\chi} \ddot{\chi} + M_{\chi\psi} \ddot{\psi} + C_{\chi\chi} \dot{\chi} + C_{\chi\psi} \dot{\psi} + K_{\chi\chi} \dot{\chi} + K_{\chi\psi} \dot{\psi} = M_{\chi}
\end{align}

$M_{\chi}$ is an external roll disturbance moment.

$M C K$ depends on physical properties of motorcycle and Forward Velocity $V$

\item
Steering equation
\begin{align}
    M_{\psi\chi} \ddot{\chi} + M_{\psi\psi} \ddot{\psi} + C_{\psi\chi} \dot{\chi} + C_{\psi\psi} \dot{\psi} + K_{\psi\chi} \dot{\chi} + K_{\psi\psi} \dot{\psi} = M_{\psi}
\end{align}

\end{enumerate}

\subsection{Expansion}
    \begin{equation}
        M_{\chi\chi} = T_{yy}
    \end{equation}

\subsection{Constitutional Relationship}
\begin{equation}
    T_{yy} = I_{yy_{r}} + m_r h^2_{r} + F_{yy'} + m_fh_f^2
\end{equation}

\begin{equation}
    F'_{yy} = I_{yy_f}\cos^2\lambda + I_{yz_f}\sin2\lambda + I_{zz_f}\sin^2\lambda
\end{equation}

\begin{equation}
    T_{zz} = I_{zz_{r}} + m_r l^2_{r} + F_{zz'} + m_f(c_w + l_f)^2
\end{equation}

\subsection{Terms determination}
\begin{equation}
    M_{\chi\chi} = T_{yy} =  m_r h^2_r + m_fh^2_f + I_{yy_r} + I_{yy_f}\cos^2\lambda + I_{yz_f}\sin2\lambda + I_{zz_f}\sin^2\lambda
\end{equation}

\section{Transfer Function}

{\em SISO (Single Input Single Output)}

Control input:
$M_{\psi}$ is steering torque applied to the front assembly by the steering motor.

Applying {\em \href{http://tutorial.math.lamar.edu/pdf/Laplace_Table.pdf}{laplace transform}} to obtain transfer function:

\section{Electric Motor Modeling}

Assume that we are using {\em DC Brushless Motor}:

\href{http://ctms.engin.umich.edu/CTMS/index.php?example=MotorSpeed&section=SystemModeling#3}{Electrical and dynamic equation}

TODO:

\begin{equation}
    T(t) = k(t) \cdot i(t)
\end{equation}

\begin{equation}
    e_{m}(t) = k_{b} \cdot \omega_{m}(t)
\end{equation}

\begin{equation}
    V_{in}(t) - e_{m}(t) = R \cdot i(t)
\end{equation}

\begin{equation}
    J_{m}\frac{d}{dt}\omega(t) = T(t) - T_{L}(t)
\end{equation}

Parameter list:
\begin{enumerate}
\item  $T(t)$ is the {\em torque} generated by the {\em armature}

\item $k(t)$ is the motor torque constant,
\item $\omega_{m}(t)$ is the armature angular velocity,
\item $e_{m}(t)$ is the back electro-magnetic force
generated by the armature spinning at an angular velocity $\omega_{m}(t)$
\item $V_{in}(t)$ is the supply voltage, $i(t)$ is the current in the armature wire
\item $T_{L}(t)$ is the load torque.
\item $J_{m}$ is armature inertia
\end{enumerate}


For an unmanned motorcycle, unless a steering torque disturbance is introduced,

\begin{equation}
    T_{L}(t) = M_{\psi}(t)
\end{equation}

% Trying to formalize in state space form

% Input:

% % \begin{equation}
% %     A=\begin{bmatrix}
% %         X_{t_{k}} &   Y_{t_{k}} & \dot{X}_{t_{k}}  &   \dot{Y}_{t_{k}}
% %         \end{bmatrix}^{T}
% %   \end{equation}
%  \begin{equation}
%      \vec{u}^{\,} = \begin{bmatrix}
%         %  X_{t_{k}} \\
%         %  Y_{t_{k}} \\
%         %  \dot{X}_{t_{k}}\\
%         %  \dot{Y}_{t_{k}}
%         M_{\chi} \\
%         M_{\psi}
%         \end{bmatrix}
% \end{equation}

\section{Symbol and parameter list}
Parameters of steering DC motor
TODO: Measurement
TODO: Uncertainty
\begin{itemize}
\item $V$ Vehilce Linear Velocity $m/s$
\item $R$ Armature Resistence $\Omega$
\item $L$ Armature Inductance $mH$
\item $k_{b}$ Back EMF Constant $\frac{V \cdot s}{rad}$
\item $k_{t}$ Torque Constant $\frac{Nm}{A}$
\item $N$ Transmission Reduction (motor gearhead and timing pulleys combined)
\end{itemize}
Motorcycle Geometric Parameter
\begin{itemize}
\item $c_{f}$ Mechanical Trail (See wiki Bicycle and Motorcycle Geometry for definition) $mm$
\item $c_{w}$ Wheel Base $m$
\item $d$ Perpendicular distance from the Steering
Axis to the Front Assembly's center of
mass $mm$
\item $h_{f}$ Height of the Front Assembly center of
mass relative to the Front Wheel contact
point $m$
\item $h_{f}$ Height of the Rear Assembly center of
mass relative to the Rear Wheel contact
\item $l_{f}$ Forward distance of the front assembly
center of mass relative to the front $m$
\item $l_{r}$ Forward distance of the rear assembly center of mass relavitve to the rear $m$
point $m$
\item $r_{f}$ Front Wheel Radius $m$
\item $r_{r}$ Rear Wheel Radius $m$
\item $\lambda$ Steering head angle (Rake angle) $degree$
\end{itemize}

Inertia and Mass
\begin{itemize}
\item $m_{f}$ Mass of Front Assembly $m$
\item $m_{r}$ Mass of Rear Assembly $m$

\item $I_{xx_{2}}$ Rear wheel moment of inertia measured about wheel axle $kg\cdot m^{2}$

\item $I_{yy_{f}}$ $y'-y'$  moment of inertia of the front
assembly, measured about front center
of mass $kg\cdot m^{2}$

\item $I_{yz_{f}}$ $y'-z'$  product of inertia of the front
assembly, measured about front center
of mass $kg\cdot m^{2}$

\item $I_{zz_{f}}$ $z'-z'$  moment of inertia of the front
assembly, measured about front center
of mass $kg\cdot m^{2}$

\item $I_{yy_{r}}$ $y'-y'$  moment of inertia of the rear
assembly, measured about front center
of mass $kg\cdot m^{2}$

\item $I_{yz_{r}}$ $y'-z'$  product of inertia of the rear
assembly, measured about front center
of mass $kg\cdot m^{2}$

\item $I_{zz_{r}}$ $z'-z'$  moment of inertia of the rear
assembly, measured about front center
of mass $kg\cdot m^{2}$

\end{itemize}

\section{Transfer function}
\begin{equation}
    \frac{\dot{\chi}}{V_{in}(s)} = - \bigg[\frac{k_{t}/R \cdot N \cdot s \cdot P_{2}(s)}{(\tau_{e}s + 1) \Big(J_{m}N^2s^2p_{1}(s) + p_{5}(s)\Big) + \Big(k_{t}k_{b}/R\Big)N^2s \cdot p_{1}(s)}\bigg]
\end{equation}

where $\tau_{e} = \frac{L}{R}$

\section{Controller Design}
Both roll angle and roll rate were measured.

Cascade Control, closing inner loop on the roll rate and an outer loop on the roll angle

Roll angle signal sampling rate?

Different outer loop control strategy. For speed?

\subsection{Notes on controller design}
Motorcycle parameters are velocity dependent. TODO: measurement of the velocity. Gain-scheduling approach. Sharp.

\subsection{Controller Design approaches comparison}
TODO: go through UMich

Linear:
QFT: Robust control with QFT
\url{http://www.math.kth.se/optsyst/forskning/forskarutbildning/5B5782/}


Control Tutorial for Matlab and Simulink
url{http://ctms.engin.umich.edu/CTMS/index.php?aux=Home}

Stephen Boydd's Linear Controller Design
LQR/LQG



Non Linear

LPV?
Dynamic Inversion?
\section{References}
\begin{itemize}
\item \url{http://tutorial.math.lamar.edu/pdf/Laplace_Table.pdf}
\item \url{http://ctms.engin.umich.edu/CTMS/index.php?example=MotorSpeed&section=SystemModeling#3}
\item \url{http://electronics.stackexchange.com/questions/33315/understanding-motor-constants-kt-and-kemf-for-comparing-brushless-dc-motors}
\item \url{http://lancet.mit.edu/motors/motors3.html}
\item \url{https://en.wikipedia.org/wiki/Motor_constants}
\item \url{https://en.wikipedia.org/wiki/Bicycle_and_motorcycle_geometry}
\item \url{http://www.math.kth.se/optsyst/forskning/forskarutbildning/5B5782/}
\end{itemize}
\end{document}
